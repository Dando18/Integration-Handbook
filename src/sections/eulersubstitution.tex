\section{Euler Substitution} \label{sec:eulersubstitution}


Leonhard Euler, a Swiss mathematician from the 1700s first introduced this method of integration, hence the name. Like reflection substitution, it's just a special case of a $u$-substitution. 

\subsection*{Definition}

Let $R(f, g)$ denote a rational function of $f$ and $g$. Given the integral,

$$ \int R(x, \sqrt{ax^2+bx+c})\,\dx x, $$

\noindent we make one of three substitutions:

1. If $a>0$, then we make the substitution,

$$ \sqrt{ax^2 + bx + c} = \pm x\sqrt a + t $$
$$ \Downarrow $$
$$ x = \frac{t^2 - c}{b + 2t\sqrt a} $$

2. If $c>0$, then we make the substitution,

$$ \sqrt{ax^2 + bx + c} = xt \pm \sqrt{c} $$
$$ \Downarrow $$
$$ x = \frac{2t\sqrt c + b}{t^2-a} $$

3. If $ax^2 + bx + c$ is has two real roots $\alpha$ and $\beta$, then we substitute,

$$ \sqrt{ax^2 + bx + c} = (x-\alpha)t $$
$$ \Downarrow $$
$$ x = \frac{a\beta - \alpha t^2}{a-t^2} $$

\subsection*{Intuition}

The goal of an Euler substitution is to eliminate a radical in a rational function. We see in the first two substitutions the $ax^2$ terms are canceled out so we don't have any $\sqrt t$ terms left over. In all three substitutions we are left with

$$ \int R(P_1(x), P_2(x))\,\dx x, $$

\noindent where $P(x)$ is a polynomial of $x$. This should be trivial to integrate using long division, partial fraction decomposition, etc. 

\subsection*{Examples}

1. Evaluate $\displaystyle \int \frac{\dx x}{\sqrt{x^2 + x + 1}}$.

\begin{ExampleBody}
    
    Since $a>0$, we make the first Euler substitution and solve for $x$, $$\sqrt{x^2 + x + 1} = x + t \implies x = \frac{t^2 - 1}{1-2t}.$$
    
    \noindent Substitute the value for $x$ back in,
    
    $$ \sqrt{x^2 + x + 1} = \frac{t^2 - 1}{1-2t} + t = \frac{-t^2+t-1}{1-2t} $$
    
    \noindent Now find the differential,
    
    $$ dx = \frac{2(-t^2+t-1)}{(1-2t)^2} dt. $$
    
    \noindent Finally we substitute these values into our integral,
    
    \begin{align*} 
    \int \frac{1-2t}{-t^2+t-1} \frac{2(-t^2+t-1)}{(1-2t)^2}\,\dx t &= \int \frac{2}{1-2t}\,\dx t \\
    &= - \ln \left| 1-2t \right| + c \\
    &= -\ln\left| 1-2\left(\sqrt{x^2+x+1}-x\right)\right| + c.
    \end{align*}
    
\end{ExampleBody}

2. Evaluate $\displaystyle \int \frac{\dx x}{\sqrt{x^2-3x-10}}$.
\begin{ExampleBody}
    
    First note that $x^2-3x-10 = (x-5)(x+2)$, which indicates we'll be using our third substitution. Substitute $\sqrt{x^2-3x-10} = (x-5)t$ and solve for $x$,
    
    $$ x = \frac{2+5t^2}{t^2-1} \implies \sqrt{x^2-3x-10} = \left(\frac{2+5t^2}{t^2-1}-5\right)t = \frac{7t}{t^2-1} $$
    
    \noindent and the differential
    
    $$ dx = \frac{-14t}{(t^2-1)^2} dt. $$
    
    \noindent Finally we substitute back in and integrate,
    
    \begin{align}
         \int \frac{t^2-1}{7t} \frac{-14t}{(t^2-1)^2}dt &= -2 \int \frac{dt}{t^2-1} \nonumber \\
        &= \ln\left| \frac{t+1}{t-1} \right| +c \nonumber \\ 
        &= \ln\left| \frac{\sqrt{x^2-3x-10} + x - 5}{\sqrt{x^2-3x-10} - x + 5} \right| + c \label{eq:eulersub1} \\
        &= \ln\left| 2x-3+2\sqrt{x^2-3x-10} \right| + c. \label{eq:eulersub2}
    \end{align}
    
    From step (\ref{eq:eulersub1}) to (\ref{eq:eulersub2}) we rationalize the denominator and absorb the $-\ln 7$ term into the constant $c$. Euler's substitution has allowed us to remove  the radical and $x^2$ term, making the integrand feasible. 
\end{ExampleBody}

\subsection*{Exercises}

\begin{enumerate}
    \item Evaluate $\displaystyle \int \frac{\dx x}{\sqrt{x^2+c}}$ for arbitrary $c$. How does this compare to when $c=\pm 1$?
    \item Evaluate $\displaystyle \int_1^2 \frac{\dx x}{x+\sqrt{x^2+1}}$.
    \item Evaluate $\displaystyle \int \frac{\dx x}{x\sqrt{x^2+x+1}}$.
\end{enumerate}

