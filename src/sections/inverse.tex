\section{Integrals of Inverse Functions} \label{sec:inverse}

First by Laisant in 1905, this formula is capable of allowing quick integration of inverse formulas.

\subsection*{Definition}

Given a bijective map $f : A \to B$, where $A, B \in \mathbb R$, the integral of the inverse $f^{-1} : B \to A$ is given by,

$$ \int f^{-1}(x)\,\dx x = xf^{-1}(x)-(F\circ f^{-1})(x) + c, $$

\noindent where $F$ is the anti-derivative of $f$.

\subsection*{Intuition}

It is trivial to show the equality if we differentiate both sides,

\begin{align*}
    f^{-1}(x) &= (f^{-1}(x) + x(f^{-1})'(x)) - F'(f^{-1}(x))(f^{-1})'(x) \\
    &= f^{-1}(x) + x(f^{-1})'(x) - x(f^{-1})'(x) \\
    &= f^{-1}(x).
\end{align*}

This is fine for showing why the identity is true, but there's an even better geometric representation. Assume that $f^{-1}$ is differentiable. Let $f(a)=c$ and $f(b)=d$. Then we can rewrite the identity as,

$$ \int_c^d f^{-1}(x)\,\dx x + \int_a^b f(x)\,\dx x = bd-ac. $$

We can now trivially see that this holds from Figure (\ref{fig:inverseplot}).

\begin{figure}
    \begin{tikzpicture}
        \begin{axis}[
            axis lines = left,
            ymax = 4.75,
            ymin = 0,
            xmin = 2.25,
            xmax = 5.5,
            ytick = {1,4},
            yticklabels={$c$,$d$},
            xtick = {3,5.25},
            xticklabels={$a$,$b$},
            xlabel = $x$,
            ylabel = {$f(x)$},
        ]
        % yes please
        \addplot[
            name path = f,
            domain = 3:5.25,
            samples = 200,
            color = red,
        ]{(x-4)^3+2};
        \addlegendentry{$f(x)$}
        \addplot[
            draw = none,
            fill = red!25,
            opacity = 0.25,
            domain = 3:5.25,
        ] {(x-4)^3+2} \closedcycle;
        
        
        \draw[thick, draw=blue] (axis cs:5.25,0) -- (axis cs:5.25,4);
        
        \draw[name path = t, thick, draw=blue] (axis cs:0,4) -- (axis cs:5.25,4);
        
        \draw[thick, draw=green] (axis cs:3,0) -- (axis cs:3,1);
        
        \draw[name path = d, thick, draw=green] (axis cs:0,1) -- (axis cs:3,1);
        
        \draw[name path = L, thick, draw=red] (axis cs:0,0) -- (axis cs:0,4);
        
        
        
        \addplot[
            color = blue!15,
        ]fill between[of=f and t, soft clip={domain=3:5.25}];
        \addplot[
            color = blue!15,
        ]fill between[of=t and d, soft clip={domain=0:3.1}];
        
        \node[draw=none] at (axis cs:3.5,3) {$\displaystyle\int_c^d f^{-1}(y)\,\dx y$};
        \node[draw=none] at (axis cs:4.25,0.75) {$\displaystyle\int_a^b f(x)\,\dx x$};
        
        \end{axis}
    
    \end{tikzpicture}
    \caption{Graphical intuition behind inverse derivative.}
    \label{fig:inverseplot}
\end{figure}

\subsection*{Examples}

1. $\displaystyle\int \arcsin x\,\dx x$

\begin{ExampleBody}
    First, notice that the function of the integrand is an inverse function, where $f^{-1}(x)=\arcsin x$ and $f(x)=\sin x$. According to the identity, we need to also find $F$, the anti-derivative of $f$, which in this case $F(x)=-\cos x$.
    
    \begin{align}
        \int \arcsin x\,\dx x &= x\arcsin x - (-\cos \circ \arcsin)(x) + c \nonumber \\
        &= x\arcsin x + \cos \arcsin x + c \nonumber \\
        &= x\arcsin x + \sqrt{1-x^2} + c \label{eq:cosarcsin}
    \end{align}
    
    In step (\ref{eq:cosarcsin}) we use the identity that $\cos \arcsin x = \sqrt{1-x^2}$ (think of the geometric meanings of the trig functions).
\end{ExampleBody}

2. $\displaystyle \int \ln x\,\dx x$

\begin{ExampleBody}
    Again, we can simply plug into the identity,
    
    \begin{align*}
        \int \ln x\,\dx x &= x\ln x - (\exp \circ \ln)(x) + c \\
        &= x(\ln x - 1) + c
    \end{align*}
\end{ExampleBody}

\subsection*{Exercises}
Evaluate the following integrals using the inverse identity of integration.

\begin{enumerate}
    \item $\displaystyle \int W(z)\,\dx z$ where $W(z)$, the W-Lambert function, is the inverse of $ze^z$.
    \item $\displaystyle \int\arctan x\,\dx x$.
    \item $\displaystyle \int \arccos x \,\dx x$.
\end{enumerate}

